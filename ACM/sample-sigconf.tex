%%
%% This is file `sample-sigconf.tex',
%% generated with the docstrip utility.
%%
%% The original source files were:
%%
%% samples.dtx  (with options: `sigconf')
%%
%% IMPORTANT NOTICE:
%%
%% For the copyright see the source file.
%%
%% Any modified versions of this file must be renamed
%% with new filenames distinct from sample-sigconf.tex.
%%
%% For distribution of the original source see the terms
%% for copying and modification in the file samples.dtx.
%%
%% This generated file may be distributed as long as the
%% original source files, as listed above, are part of the
%% same distribution. (The sources need not necessarily be
%% in the same archive or directory.)
%%
%% The first command in your LaTeX source must be the \documentclass command.
\documentclass[sigconf,anonymous]{acmart}
\usepackage{diagbox}
\usepackage{multirow}
\newcommand{\dou}[1]{\textcolor{red}{[[@dou: #1]]}}
\newcommand{\jing}[1]{\textcolor{blue}{[[@jing: #1]]}}
\newcommand{\pt}[1]{\textcolor{green}{[[@pei: #1]]}}

%%
%% \BibTeX command to typeset BibTeX logo in the docs
\AtBeginDocument{%
  \providecommand\BibTeX{{%
    \normalfont B\kern-0.5em{\scshape i\kern-0.25em b}\kern-0.8em\TeX}}}

%% Rights management information.  This information is sent to you
%% when you complete the rights form.  These commands have SAMPLE
%% values in them; it is your responsibility as an author to replace
%% the commands and values with those provided to you when you
%% complete the rights form.
\setcopyright{acmcopyright}
\copyrightyear{2021}
\acmYear{2021}
\acmDOI{10.1145/1122445.1122456}

%% These commands are for a PROCEEDINGS abstract or paper.
% \acmConference[WSDM '22]{15th ACM International WSDM Conference}{February 21--25, 2022}{Phoenix, Arizona}
% \acmBooktitle{WSDM '22: 15th ACM International Conference on Web Search and Data Mining, February 21--25, 2022, Phoenix, Arizona}
\acmPrice{15.00}
\acmISBN{978-1-4503-XXXX-X/18/06}

%%
%% Submission ID.
%% Use this when submitting an article to a sponsored event. You'll
%% receive a unique submission ID from the organizers
%% of the event, and this ID should be used as the parameter to this command.
%%\acmSubmissionID{123-A56-BU3}

%%
%% The majority of ACM publications use numbered citations and
%% references.  The command \citestyle{authoryear} switches to the
%% "author year" style.
%%
%% If you are preparing content for an event
%% sponsored by ACM SIGGRAPH, you must use the "author year" style of
%% citations and references.
%% Uncommenting
%% the next command will enable that style.
%%\citestyle{acmauthoryear}

%%
%% end of the preamble, start of the body of the document source.


\begin{document}

%%
%% The "title" command has an optional parameter,
%% allowing the author to define a "short title" to be used in page headers.
\title{Towards Efficiency and Sparsity: Exploiting Personalized Terms for News Recommendation}

%%
%% The "author" command and its associated commands are used to define
%% the authors and their affiliations.
%% Of note is the shared affiliation of the first two authors, and the
%% "authornote" and "authornotemark" commands
%% used to denote shared contribution to the research.

% \author{Peitian Zhang}
% \email{zpt@ruc.edu.cn}
% % \orcid{1234-5678-9012}
% \affiliation{%
%   \institution{Renmin University of China}
%   \streetaddress{}
%   \city{Beijing}
%   \state{}
%   \country{China}
%   \postcode{100872}
% }
% \author{Zhicheng Dou}
% \email{dou@ruc.edu.cn}
% % \orcid{1234-5678-9012}
% \affiliation{%
%   \institution{Renmin University of China}
%   \streetaddress{}
%   \city{Beijing}
%   \state{}
%   \country{China}
%   \postcode{100872}
% }

% \author{Jing Yao}
% \email{jing@ruc.edu.cn}
% % \orcid{1234-5678-9012}
% \affiliation{%
%   \institution{Renmin University of China}
%   \streetaddress{}
%   \city{Beijing}
%   \state{}
%   \country{China}
%   \postcode{100872}
% }


%%
%% By default, the full list of authors will be used in the page
%% headers. Often, this list is too long, and will overlap
%% other information printed in the page headers. This command allows
%% the author to define a more concise list
%% of authors' names for this purpose.
% \renewcommand{\shortauthors}{Trovato and Tobin, et al.}

%%
%% The abstract is a short summary of the work to be presented in the
%% article.
% \begin{abstract}
%     Some terms in a user's browsed news reveal her interests from the most fine-grained level. Capturing these \textbf{personalized terms} yields a highly compact and accurate user profile, significantly reducing the cost of using effective yet expensive matching models (e.g.BERT). The sparse terms bring interpretability to recommenders: the candidate generation process can be aligned to ad-hoc retrieval if we consider the personalized terms as query and the whole news set as document collection, where fast retrieval techniques can be employed. Due to lack of ground truth labels, we empower the model to learn to select representative terms, gaining a consistent improvement over heuristic selection baselines.
% \end{abstract}

%%
%% The code below is generated by the tool at http://dl.acm.org/ccs.cfm.
%% Please copy and paste the code instead of the example below.
% %%
% \begin{CCSXML}
% <ccs2012>
%  <concept>
%   <concept_id>10010520.10010553.10010562</concept_id>
%   <concept_desc>Computer systems organization~Embedded systems</concept_desc>
%   <concept_significance>500</concept_significance>
%  </concept>
%  <concept>
%   <concept_id>10010520.10010575.10010755</concept_id>
%   <concept_desc>Computer systems organization~Redundancy</concept_desc>
%   <concept_significance>300</concept_significance>
%  </concept>
%  <concept>
%   <concept_id>10010520.10010553.10010554</concept_id>
%   <concept_desc>Computer systems organization~Robotics</concept_desc>
%   <concept_significance>100</concept_significance>
%  </concept>
%  <concept>
%   <concept_id>10003033.10003083.10003095</concept_id>
%   <concept_desc>Networks~Network reliability</concept_desc>
%   <concept_significance>100</concept_significance>
%  </concept>
% </ccs2012>
% \end{CCSXML}

% \ccsdesc[500]{Computer systems organization~Embedded systems}
% \ccsdesc[300]{Computer systems organization~Redundancy}
% \ccsdesc{Computer systems organization~Robotics}
% \ccsdesc[100]{Networks~Network reliability}

%%
%% Keywords. The author(s) should pick words that accurately describe
%% the work being presented. Separate the keywords with commas.
\keywords{}

%% A "teaser" image appears between the author and affiliation
%% information and the body of the document, and typically spans the
%% page.
% \begin{teaserfigure}
%   \includegraphics[width=\textwidth]{sampleteaser}
%   \caption{Seattle Mariners at Spring Training, 2010.}
%   \Description{Enjoying the baseball game from the third-base
%   seats. Ichiro Suzuki preparing to bat.}
%   \label{fig:teaser}
% \end{teaserfigure}
\settopmatter{printacmref=false}
%%
%% This command processes the author and affiliation and title
%% information and builds the first part of the formatted document.
\maketitle

\section{Introduction}
\label{section:introduction}
% news recommendation is important
With the ever growing data and improving technologies, more and more news overwhelms people with its countless topics and events. In this information torrent, an individual is nearly impossible to pick out small drops that she's interested in. Luckily, online news platforms e.g. MSN\footnote{https://www.msn.com/en-us/news} serve that function for each user, significantly alleviating the information overload problem~\cite{personalized_news_recommendation}. The core technology of such platforms is personlaized news recommendation.

% news recommender should be efficient
News recommendation comprises of two major steps: recall and rerank: the former recalls a small subset of news that is relevant to the user's preference from the entire news collection; the latter reranks this subset in the descending order of click probability. In both stages, efficiency is the top concern~\cite{liu_Hifi-ark,LightRec,Octopus}.

The mainstream candidate recall methods rely on similarity search: they represent users and news articles as vectors in the same semantic space, where the top $K$ nearest neighbors of a user representation are retrieved as relevant candidates. In IR, the sparse counterpart against this dense method, which retrieves documents with exact match of the query by inverted index, is even more efficient~\cite{sparse_and_dense_recall}. It also provides more interpretable and controlled patterns that appeals to the industry~\cite{COIL}. However, lacking of an explicit query hinders the implementation of sparse recall.
% QUES: related work? how to say their deficiencies?
Early works labels each news with some pre-defined tags or keywords, then generate a user profile by collecting tags of her browsed news. However, it's always difficult to design tags at proper granularity that is precise enough to match relevant candidates and low-cost enough to conduct and maintain. Another concern of these methods is personalization. Different users may attend to different aspects of news~\cite{liu_Octopus}, however, the keywords of one piece of news is the same regardless of users. Recent methods exploit heuristics ~\cite{BM25} or neural networks~\cite{} to extract keywords from historical news text, which improves precision and avoids laborous engineering. Yet the gap to personalization remains.
% However, due to lack of explicit queries, it is a tough challenge to implement sparse recall in recommendation scenario.

When it comes to rerank stage, two-tower structure is the most popular. It encodes each news article and each user into a compact vector respectively, then computes the similarity between them as the click probability. Under this setting, various news/user encoder has been explored, such as RNN~\cite{lian_DFM}, CNN~\cite{wu_NAML}, and a number of attentive methods~\cite{wu_NRMS, NPA, NAML}.
% Recently, large-scale pretrained language models (PLM in short) e.g. BERT~\cite{Bert} greatly outperforms shallow and light-weighted neural models in NLP field, which implies their potential to improve news recommendation accuracy. Under the two-tower setting,
Recently, Wu et al.~\cite{wu_newsPLM} integrates PLMs to be the news encoder and validates their effectiveness. Despite improving accuracy, PLMs consume much more time and space when encoding due to their deep and large-scale structure. This problem is even worse in news recommendation: the model must encode all of the user's historical news and every candidate news for a single impression, which may contain thousands of words in total. In consequence, existing methods are forced to encode each historical news then feed a pooled representation  to the user encoder. In this workflow, fine-grained interactions across different news are lost, which is proved to be useful in IR.~\cite{PolyEncoder}

% why select personalized terms?


% PLMs can improve effectiveness


% Efforts that have been made


% dynamic topics


\section{Related Work}
\label{section:related work}
In this section, we review the related work of news recommendation, feature selection and candidate recall.
\subsection{News Recommendation}
% traditional news recommendation
News recommendation has been widely explored for decades. Traditional collaborative filtering methods~\cite{das_CF, li_SCENE} hash similar users into the same group by LSH before recommending. They either employ Matrix Factorization~\cite{koren_MF} to gather similar users. MF decomposes the user-item matrix to map users and items into the same latent space, where the inner product between their vectors captures the interaction score. Unlike movies or products, enoumous news is spawn every second, outstriping the increase of users, and makes the user-item matrix especially sparse. Factorization Machine~\cite{rendle_FM} then introduces real-valued features to MF to alleviate the sparsity, but it requires manual features which is time and labor demanding.

% content-based recommendation and two tower workflow
Content-based news recommendation then emerges to address the above issues. Early works of content-based methods still rely on some manual features such as trend~\cite{liu_bayesian_news_trend}, geographics~\cite{li_contextual_bandit}, and demographics~\cite{cheng_Wide&Deep} to model news and users. In the recent ten years, deep learning techniques prosper and greatly promote the progress of news recommendation systems. More and more works are underway to design equisite structures to learn representation of news and users directly from raw texts and browsing history respectively, taking the dot product between them as the click probability. This \emph{two tower} workflow appeals to industry since encoding of news and user can be done offline, greatly improving model efficiency. Under the two tower setting, Wu et al.~\cite{wu_GNN,wu_heterogeneous,wu_LSTUR,wu_NAML,wu_NPA,wu_NRMS,wu_topic-aware} proposes effective models that employs CNN, RNN, multi-head attention, and personalized attention to represent news and users. External information such as knowledge~\cite{wang_DKN} and user-item biparititie graph~\cite{wu_GNN,hu_GNN_disentanglement} are also incorporated to enhance representations.

\subsection{Efficient Transformers}
% model-level optimization for efficient transformers
More recently, large-scale pretrained language models (PLM in short) e.g. BERT~\cite{Bert} demonstrates impressive improvements over shallow and light-weighted neural models in NLP field. Though a previous work~\cite{wu_newsPLM} integrates PLMs to news recommendation and validates their improvements, it is yet non-trivial to efficiently implement them: the quadratic complexity of self-attention w.r.t. the input sequence length poses an intense challenge to speed up encoding users with dozens of historical articles that may contain thousands of words in total. Researchers make a lot of efforts to lessen this problem. SpeedyFeed~\cite{xiao_speedy} deduplicates user's historical news and candidate news in a batch, significantly speeding up the training process. Apart from optimizing training scheme in industry, more research modifies the self-attention layer to reduce complexity. For example, Longfomer~\cite{Longformer} sparsifies the full self-attention to a sliding and dialated pattern, which reduces the complexity to linear w.r.t. the input length; Fastformer achieves the same result by additive attention and element-wise production. It also achieves a new state-of-the-art performance on MIND~\cite{wu_MIND}, a large-scale dataset in news recommendation.

% input level optimization for efficient transformers
Another line of research followes a feature selection intuition that prunes the input before expensive interaction. Hofstätter et al.~\cite{Intra-Document-Cascading}
restricts BERT to only inspect top $K$ important passages per document. It splits the selection and ranking stage, where the former is trained in teacher-student paradiagm by the pseudo labels produced by a BERT, and the latter only scores the selected passages. Using the similar cascading setting, \citet{ali_SIM} extracts fewer valueble items from user history to feed into final ranking. However, cascading achitechture requires labels for each stage, which is prohibitive in our scenario because there is no ground-truth indicating terms that the user really favors. Gallagher et al.~\cite{gallagher_joint_cascade} explores framework for jointly optimizing cascade search, but it is not practical in a BERT-style model since it relies on specified empirical risk rather than ranking loss. The first application of sparse selection in news recommendation is SFI~\cite{zpt}. It sparsely and automatically selects important historical news before effective candidate-aware interaction, guaranteeing the efficiency and effectiveness of the model. Despite its improvements, SFI executes selection at new level, possibly losing information hidden in the unselected news for the final ranking. It also neglects PLMs to promote performance due to its one tower limitation.

In our work, we select the browsing history at word-level to keep more fine-grained and comprehensive information instead of pruning the historical news set. With only a handful of personlaized terms, we apply PLMs to fully capture the interaction within and among historical articles, promoting the effectiveness of the news recommender to a new level at a competitive speed.

\subsection{Candidate Recall}
The mainstream candidate recall methods rely on similarity search: they project users and items into the same latent space, where the top $K$ nearest neighbors of a user are retrieved as relevant items. In IR, the sparse counterpart of such dense recall, which retrieves documents with exact match of the query, is even more efficient~\cite{sparse_and_dense_recall}. However, there is no explicit query in recommendation, prohibiting the use of sparse recall techniques.

In our work, the generated personalized terms natually compresses the entire user history into a rather short query, with which we can align the candidate recall to ad-hoc retrieval and take advantage of sparse recall techniques. Due to space limitations, we only explore the naive sparse recall with personalized terms in this paper, and leave other advanced methods such as COIL~\cite{COIL} to future works.

% retrieval techniques organize documents by inverted-index, and sort them by their frequency-based scores such as BM25~\cite{BM25} and query likelihood~\cite{query_likelihood} given a query. However, they face two main problems: non-contexts: frequency-based scores ignore contexts; semantic-mismatch: only the documents with exact match could be retrieved, where out-of-vocabulary harms the retrieval performance; numerous works are proposed to address them. DeepCT~\cite{DeepCT} and HDCT~\cite{HDCT} compute contextualized scores instead of BM25. COIL~\cite{COIL} and SNRM~\cite{SNRM} learn semantic-aware representation for each word, while keeping the sparsity to combine inverted index. Some works~\cite{Doc2Query,SparTerm,DeepImpact} expand documents and quries for wider reception field.




\section{Experiment}
\subsection{Results}

\begin{table}
  \centering
  \begin{tabular}{lp{0.2\linewidth}cccp{0.1\linewidth}}
    \toprule
    \textbf{Type} & \textbf{Methods}&\textbf{AUC}&\textbf{MRR}&\textbf{NDCG@5}&\textbf{NDCG@10}\\
    \midrule
    \multirow{1}{0.15\linewidth}{Sota}
    &FastFormer & $\mathbf{72.68}$ & $\mathbf{37.45}$ & $\mathbf{41.51}$ & $\mathbf{46.84}$\\
    \midrule
    \multirow{4}{0.15\linewidth}{Baselines}
    &Two Tower& ${71.43}$ & ${36.16}$ & ${39.67}$ & ${45.29}$\\
    &TES-First & ${68.00}$ & ${32.91}$ & ${36.44}$ & ${42.86}$\\
    &TES-BM25 & ${68.00}$ & ${32.91}$ & ${36.44}$ & ${42.86}$\\
    &TES-Entity & ${68.00}$ & ${32.91}$ & ${36.44}$ & ${42.86}$\\
    \midrule
    \multirow{1}{0.15\linewidth}{Ours}
    &TES & $69.62$ & $34.30$ & ${37.47}$ & ${43.21}$\\
    \bottomrule
  \end{tabular}
\end{table}
% \begin{acks}
% \end{acks}
\clearpage
%%
%% The next two lines define the bibliography style to be used, and
%% the bibliography file.
\bibliographystyle{ACM-Reference-Format}
\bibliography{sample-base}

%%
%% If your work has an appendix, this is the place to put it.

\end{document}
\endinput
%%
%% End of file `sample-sigconf.tex'.